\documentclass[aps,prb,onecolumn,notitlepage,showpacs,floatfix,superscriptaddress]{revtex4-1}
\usepackage{dcolumn}
\usepackage{tabularx}
\usepackage{bm}
\usepackage{soul}
\usepackage{amsmath,xcolor}
\fboxrule=1pt
\usepackage{amssymb,graphicx}
\usepackage[colorlinks=true,citecolor=blue,urlcolor=blue,linkcolor=blue]{hyperref}
\usepackage{environ}

\usepackage{tikz}
\usetikzlibrary{matrix}
\usetikzlibrary{fit}

\NewEnviron{eqnsplit}{%
\begin{equation}
\begin{split}
  \BODY
\end{split}
\end{equation}
}
\newcommand{\mrm}[1]{\mathrm{#1}}
\newcommand{\AR}[1]{\textcolor{red}{#1}}
\newcommand{\ang}{\mathrm{\AA}}

\bibliographystyle{apsrev4-1}

%%%%%%%%%%%%%%%%%%%%%%%%%%%%%%%%%%%%%%%%%%%%%%%%
\begin{document}
\title{Open Quantum Systems: Microscopic Origin}

\author{Avinash Rustagi}
\email{arustag@purdue.edu}
%
%\date{\today}
%%%%%%%%%%%%%%%%%%%%%%%%%%%%%%%%%%%%%%%%%%%%%%%%

\maketitle
%
\noindent \textcolor{blue}{Reference: Lectures from Prof. Ivan Deutsch}
\vspace{0.2in}

As we are aware that no system is truly isolated. Thus, the quantum systems are not closed but open, meaning that there are dissipation and re-feeding mechanisms that play a role when correctly describing a physical system. In this note, we would like to understand the dynamical equations from a microscopic point of view. \\

We know that quantum mechanics is the theory that correctly describes the dynamics of a system and this dynamics is governed by the Hamiltonian. To correctly describe the dynamics, we can divide the entire Hilbert space system into system + environment. Here, our Hamiltonian will have three parts: system Hamiltonian, environment Hamiltonian, and the interaction between the system and environment. This interaction term is particularly important as it is what makes the system open i.e. dissipation.
\begin{equation}
\mathcal{H} = \mathcal{H}_S + \mathcal{H}_E + \mathcal{H}_{int}
\end{equation}
Thus, the total system can be described by a density matrix $\rho_{SE}$ that contains information about both the system and environment. \\

To simplify the calculations, we choose to work in the interaction picture where the dynamics of the system is governed by the interaction term and the trivial dynamics arising from $\mathcal{H}_S + \mathcal{H}_E$ is explicitly taken care of. In the interaction picture,
\begin{equation}
\rho_{SE} (t+\Delta t) = \rho_{SE} (t) - \dfrac{i}{\hbar} \int_t^{t+\Delta t} dt^\prime \, [\mathcal{H}_{int}(t^\prime), \rho_{SE} (t^\prime)]
\end{equation}
We can write the solution to this equation iteratively and upto second order,
\begin{equation}
\rho_{SE} (t+\Delta t) = \rho_{SE} (t) - \dfrac{i}{\hbar} \int_t^{t+\Delta t} dt^\prime \, [\mathcal{H}_{int}(t^\prime), \rho_{SE} (t)] - \dfrac{1}{\hbar^2} \int_t^{t+\Delta t} dt^\prime \, \int_t^{t^\prime} dt^{\prime \prime}  \, [\mathcal{H}_{int}(t^\prime), [\mathcal{H}_{int}(t^{\prime\prime}), \rho_{SE} (t)]]
\end{equation}
The language of density matrix is particularly useful as we can extract the information/dynamics of parts of our total system by tracing out the degrees of freedom corresponding to everything else. Thus, for our system dynamics, we can trace out the environment
\begin{equation}
\rho_{S} (t) = \text{Tr}_E \{ \rho_{SE} (t)\}
\end{equation}
\end{document}
