\documentclass[aps,prb,onecolumn,notitlepage,showpacs,floatfix,superscriptaddress]{revtex4-1}
\usepackage{dcolumn}
\usepackage{tabularx}
\usepackage{bm}
\usepackage{soul}
\usepackage{amsmath,xcolor}
\fboxrule=1pt
\usepackage{amssymb,graphicx}
\usepackage[colorlinks=true,citecolor=blue,urlcolor=blue,linkcolor=blue]{hyperref}
\usepackage{environ}

\usepackage{tikz}
\usetikzlibrary{matrix}
\usetikzlibrary{fit}

\NewEnviron{eqnsplit}{%
\begin{equation}
\begin{split}
  \BODY
\end{split}
\end{equation}
}
\newcommand{\mrm}[1]{\mathrm{#1}}
\newcommand{\AR}[1]{\textcolor{red}{#1}}
\newcommand{\ang}{\mathrm{\AA}}

\bibliographystyle{apsrev4-1}

%%%%%%%%%%%%%%%%%%%%%%%%%%%%%%%%%%%%%%%%%%%%%%%%
\begin{document}
\title{Thiele Formalism: Dynamics of spin textures}

\author{Avinash Rustagi}
\email{arustag@purdue.edu}
%
%\date{\today}
%%%%%%%%%%%%%%%%%%%%%%%%%%%%%%%%%%%%%%%%%%%%%%%%

\maketitle
%
\noindent \textcolor{blue}{Reference: Phys. Rev. Lett. 30, 230 (1973)}
\vspace{0.2in}

\noindent The magnetization texture in a thin nanomagnet can be written as
\begin{equation}
\vec{m}(\vec{r},t) = \vec{m}(\vec{r}-\vec{R}(t))
\end{equation}
where $\vec{r}$ is in the plane normal to the thin film normal. Starting with the ansatz that the magnetization texture moves as a whole without changes in the texture allows us to write 
\begin{equation}
\dot{\vec{m}} = - (\dot{\vec{R}}\cdot \vec{\nabla}) \vec{m}
\end{equation}
The magnetization dynamics is governed by the LLG equation
\begin{equation}
\dot{\vec{m}} = -\gamma \, \vec{m} \times \vec{H}_\mrm{eff} + \alpha \, \vec{m} \times \dot{\vec{m}}
\end{equation}
which is equivalent to
\begin{equation}
\vec{m} \times \dot{\vec{m}} = -\gamma \, (\vec{m} \cdot \vec{H}_\mrm{eff}) \vec{m} + \gamma \vec{H}_\mrm{eff} - \alpha \, \dot{\vec{m}}
\end{equation}
Since
\begin{equation}
\vec{m} \cdot \dot{\vec{m}}= 0 \Rightarrow - \dot{R}_j \vec{m} \cdot \partial_j \vec{m} = 0
\end{equation}
and since $\dot{R}_j \neq 0$, $\vec{m} \cdot \partial_j \vec{m} = 0$.\\

Upon substitution of the Thiele ansatz into the LLG
\begin{equation}
-\vec{m} \times (\dot{\vec{R}}\cdot \vec{\nabla}) \vec{m} = -\gamma \, (\vec{m} \cdot \vec{H}_\mrm{eff}) \vec{m} + \gamma \vec{H}_\mrm{eff} + \alpha \, (\dot{\vec{R}}\cdot \vec{\nabla})\vec{m}
\end{equation}
which is equivalent to
\begin{equation}
-\dot{R}_i \, \vec{m} \times \partial_i \vec{m} = -\gamma \, (\vec{m} \cdot \vec{H}_\mrm{eff}) \vec{m} + \gamma \vec{H}_\mrm{eff} + \alpha \, \dot{R}_i \, \partial_i \vec{m}
\end{equation}
where repeated indices are summed over. Taking dot product with $\partial_x \vec{m}$ and $\partial_y \vec{m}$, we get the two equations for the magnetization texture location in the thin magnet,
\begin{equation}
\begin{split}
-\dot{R}_y \, (\vec{m} \times \partial_y \vec{m}) \cdot \partial_x \vec{m} &= \gamma \vec{H}_\mrm{eff} \cdot \partial_x \vec{m} + \alpha \, \dot{R}_i \, \partial_i \vec{m} \cdot \partial_x \vec{m} \\
-\dot{R}_x \, (\vec{m} \times \partial_x \vec{m}) \cdot \partial_y \vec{m} &= \gamma \vec{H}_\mrm{eff} \cdot \partial_y \vec{m} + \alpha \, \dot{R}_i \, \partial_i \vec{m} \cdot \partial_y \vec{m}
\end{split}
\end{equation}
Since $(\vec{m} \times \partial_y \vec{m}) \cdot \partial_x \vec{m} = - (\partial_x \vec{m} \times \partial_y \vec{m}) \cdot  \vec{m}$ and $(\vec{m} \times \partial_x \vec{m}) \cdot \partial_y \vec{m} =  (\partial_x \vec{m} \times \partial_y \vec{m}) \cdot  \vec{m}$,
\begin{equation}
\begin{split}
\dot{R}_y \, (\partial_x \vec{m} \times \partial_y \vec{m}) \cdot  \vec{m} &= \gamma \vec{H}_\mrm{eff} \cdot \partial_x \vec{m} + \alpha \, \dot{R}_i \, \partial_i \vec{m} \cdot \partial_x \vec{m} \\
-\dot{R}_x \, (\partial_x \vec{m} \times \partial_y \vec{m}) \cdot  \vec{m} &= \gamma \vec{H}_\mrm{eff} \cdot \partial_y \vec{m} + \alpha \, \dot{R}_i \, \partial_i \vec{m} \cdot \partial_y \vec{m}
\end{split}
\end{equation}
Integrating over the spatial dimension ($L\int d\vec{r}$) and analyzing terms one at a time. 
\begin{equation}
\begin{split}
\dot{R}_y L\int d\vec{r}\,(\partial_x \vec{m} \times \partial_y \vec{m}) \cdot  \vec{m} &= \gamma L\int d\vec{r}\,\vec{H}_\mrm{eff} \cdot \partial_x \vec{m} + \alpha \, \dot{R}_i L\int d\vec{r}\, \partial_i \vec{m} \cdot \partial_x \vec{m} \\
-\dot{R}_x L\int d\vec{r}\, (\partial_x \vec{m} \times \partial_y \vec{m}) \cdot  \vec{m} &= \gamma L\int d\vec{r}\,\vec{H}_\mrm{eff} \cdot \partial_y \vec{m} + \alpha \, \dot{R}_i L\int d\vec{r}\, \partial_i \vec{m} \cdot \partial_y \vec{m}
\end{split}
\end{equation}
For the first term on the right hand side, let us consider the force experienced by the magnetic texture
\begin{equation}
\begin{split}
F_j &= - \dfrac{dE}{dR_j} = \dfrac{d}{dR_j} M_s L \int d\vec{r} \, \vec{H}_\mrm{eff} \cdot \vec{m}(\vec{r}-\vec{R}) \\
&=  M_s L \int d\vec{r} \, \vec{H}_\mrm{eff} \cdot \dfrac{d \vec{m}}{dR_j} \\
&= -M_s L \int d\vec{r} \, \vec{H}_\mrm{eff} \cdot \dfrac{d \vec{m}}{dr_j} \equiv  -M_s L \int d\vec{r} \, \vec{H}_\mrm{eff} \cdot \partial_j \vec{m}\\
\end{split}
\end{equation}
since the energy density is $-M_s \vec{H}_\mrm{eff} \cdot \vec{m}(\vec{r}-\vec{R})$. Therefore,
\begin{equation}
\gamma L\int d\vec{r}\,\vec{H}_\mrm{eff} \cdot \partial_j \vec{m} = -\dfrac{\gamma}{M_s} F_j
\end{equation}
Consequently,
\begin{widetext}
\begin{equation}
\begin{split}
\dot{R}_y \dfrac{M_s L}{\gamma}\int d\vec{r}\,(\partial_x \vec{m} \times \partial_y \vec{m}) \cdot  \vec{m} &= -F_x + \alpha \dfrac{M_s L}{\gamma} \int d\vec{r}\, \partial_x \vec{m} \cdot \partial_i \vec{m} \,\, \dot{R}_i  \\
-\dot{R}_x \dfrac{M_s L}{\gamma}\int d\vec{r}\, (\partial_x \vec{m} \times \partial_y \vec{m}) \cdot  \vec{m} &=- F_y + \alpha \dfrac{M_s L}{\gamma} \int d\vec{r}\, \partial_y \vec{m} \cdot \partial_i \vec{m} \,\, \dot{R}_i  \\
\end{split}
\end{equation}
\end{widetext}
Using the definition for topological charge
\begin{equation}
Q = \dfrac{1}{4\pi} \int d\vec{r}\,(\partial_x \vec{m} \times \partial_y \vec{m})\cdot \vec{m}
\end{equation}
we can define the Gyrotropic vector
\begin{equation}
\vec{G} = \hat{z} \, \dfrac{4\pi M_s L}{\gamma} Q = G_z \, \hat{z}
\end{equation}
and defining the dissipation tensor elements as
\begin{equation}
D_{pq} = \alpha \dfrac{M_s L}{\gamma} \int d\vec{r}\, \partial_p \vec{m} \cdot \partial_q \vec{m}
\end{equation}
With these definitions, the equation of motion for the Skyrmion center is
\begin{equation}
\begin{split}
\dot{R}_y G_z &= -F_x + D_{xi}\, \dot{R}_i  \\
-\dot{R}_x G_z &= -F_y + D_{yi}\, \dot{R}_i  \\
\end{split}
\end{equation}
Therefore in vectorial form, the equation of motion is
\begin{equation}
\dot{\vec{R}} \times \vec{G} = -\vec{F} + \bar{\bar{D}} \, \dot{\vec{R}}
\end{equation}
where $\bar{\bar{D}}$ is the dissipation tensor with elements given by $D_{pq}$.
\subsection{Solution:}
We can easily solve the equation of motion under the action of a linear restoring force in absence of dissipation,
\begin{equation}
\begin{split}
\dot{R}_y G_z &= k R_x  \\
-\dot{R}_x G_z &= k R_y   \\
\end{split}
\end{equation}
Defining a complex variable $Z = R_x +i R_y$,
\begin{equation}
\begin{split}
-i\dot{Z} G_z &= k Z   \Rightarrow \dot{Z} = i \omega_G Z\\
\end{split}
\end{equation}
Laplace transforming,
\begin{equation}
Z(s) = \dfrac{Z(0)}{s-i\omega_G}
\end{equation}
Therefore,
\begin{equation}
Z(t) = Z(0) \exp\left( i\omega_G t\right)
\end{equation}
which implies
\begin{equation}
\begin{split}
R_x(t) &= R_x(0) \cos\omega_G t - R_y(0) \sin\omega_G t \\
R_y(t) &= R_x(0) \sin\omega_G t + R_y(0) \cos\omega_G t 
\end{split}
\end{equation}
\end{document}
