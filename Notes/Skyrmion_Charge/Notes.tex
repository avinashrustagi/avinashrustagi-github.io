\documentclass[aps,prb,onecolumn,notitlepage,showpacs,floatfix,superscriptaddress]{revtex4-1}
\usepackage{dcolumn}
\usepackage{tabularx}
\usepackage{bm}
\usepackage{soul}
\usepackage{amsmath,xcolor}
\fboxrule=1pt
\usepackage{amssymb,graphicx}
\usepackage[colorlinks=true,citecolor=blue,urlcolor=blue,linkcolor=blue]{hyperref}
\usepackage{environ}

\usepackage{tikz}
\usetikzlibrary{matrix}
\usetikzlibrary{fit}

\NewEnviron{eqnsplit}{%
\begin{equation}
\begin{split}
  \BODY
\end{split}
\end{equation}
}
\newcommand{\mrm}[1]{\mathrm{#1}}
\newcommand{\AR}[1]{\textcolor{red}{#1}}
\newcommand{\ang}{\mathrm{\AA}}

\bibliographystyle{apsrev4-1}

%%%%%%%%%%%%%%%%%%%%%%%%%%%%%%%%%%%%%%%%%%%%%%%%
\begin{document}
\title{Skyrmion: Topological Charge}

\author{Avinash Rustagi}
\email{arustag@purdue.edu}
%
%\date{\today}
%%%%%%%%%%%%%%%%%%%%%%%%%%%%%%%%%%%%%%%%%%%%%%%%

\maketitle
%
The topological charge of a skyrmion is defined as
\begin{equation}
Q_\mrm{sk} =\dfrac{1}{4\pi}\int dxdy\, \vec{m} \cdot \left( \dfrac{\partial \vec{m}}{\partial x} \times \dfrac{\partial \vec{m}}{\partial y} \right)
\end{equation}
The unit magnetization can be described in terms of two fields $\Theta,\Phi$:
\begin{equation}
\vec{m} = \left( \begin{array}{c}
\sin\Theta \cos\Phi \\ 
\sin\Theta \sin\Phi \\ 
\cos\Theta
\end{array} \right)
\end{equation}
Evaluating the terms in skymrion charge density
%\begin{widetext}
\begin{equation}
\begin{split}
m_x \left( \dfrac{\partial \vec{m}}{\partial x} \times \dfrac{\partial \vec{m}}{\partial y} \right)_x &= \sin^3\Theta \cos^2\Phi \left[ \dfrac{\partial \Theta}{\partial x}\dfrac{\partial \Phi}{\partial y}-\dfrac{\partial \Phi}{\partial x}\dfrac{\partial \Theta}{\partial y} \right] \\
m_y \left( \dfrac{\partial \vec{m}}{\partial x} \times \dfrac{\partial \vec{m}}{\partial y} \right)_y &= \sin^3\Theta \sin^2\Phi \left[ \dfrac{\partial \Theta}{\partial x}\dfrac{\partial \Phi}{\partial y}-\dfrac{\partial \Phi}{\partial x}\dfrac{\partial \Theta}{\partial y} \right] \\
m_z \left( \dfrac{\partial \vec{m}}{\partial x} \times \dfrac{\partial \vec{m}}{\partial y} \right)_z &= \sin\Theta \cos^2\Theta \left[ \dfrac{\partial \Theta}{\partial x}\dfrac{\partial \Phi}{\partial y}-\dfrac{\partial \Phi}{\partial x}\dfrac{\partial \Theta}{\partial y} \right] \\
\end{split}
\end{equation}
%\end{widetext}
Therefore,
\begin{equation}
Q_\mrm{sk} =\dfrac{1}{4\pi}\int dxdy\, \sin\Theta \left[ \dfrac{\partial \Theta}{\partial x}\dfrac{\partial \Phi}{\partial y}-\dfrac{\partial \Phi}{\partial x}\dfrac{\partial \Theta}{\partial y} \right]
\end{equation}
The transformation from cartesian to polar coordinates
\begin{equation}
\begin{split}
\dfrac{\partial \Theta}{\partial x} &= \cos\phi \dfrac{\partial \Theta}{\partial r} -\dfrac{\sin\phi}{r} \dfrac{\partial \Theta}{\partial \phi} \\
\dfrac{\partial \Theta}{\partial y} &= \sin\phi \dfrac{\partial \Theta}{\partial r} +\dfrac{\cos\phi}{r} \dfrac{\partial \Theta}{\partial \phi} \\
\dfrac{\partial \Phi}{\partial x} &=  \cos\phi \dfrac{\partial \Phi}{\partial r} -\dfrac{\sin\phi}{r} \dfrac{\partial \Phi}{\partial \phi}\\
\dfrac{\partial \Phi}{\partial y} &= \sin\phi \dfrac{\partial \Phi}{\partial r} +\dfrac{\cos\phi}{r} \dfrac{\partial \Phi}{\partial \phi} \\
\end{split}
\end{equation}
Invoking the observation $\Theta = \Theta(r)$ and $\Phi = \Phi(\phi)$,
\begin{equation}
\begin{split}
Q_\mrm{sk} &=\dfrac{1}{4\pi}\int_0^\infty dr \, \sin\Theta \dfrac{d\Theta}{dr} \int_0^{2\pi} d\phi\,  \dfrac{d\Phi}{d\phi} \\
&= \dfrac{[m_z(0)-m_z(\infty)]}{2} \dfrac{[\Phi(2\pi)-\Phi(0)]}{2\pi} 
\end{split}
\end{equation}
For a skyrmion with magnetization in +z direction at the center and -z direction outside the the domain wall implies $m_z(0)-m_z(\infty) = 2$, and with $\Phi(\phi) = \nu \phi + \gamma$ where $\nu$ is the vorticity and $\gamma$ is the helicity,
\begin{equation}
\begin{split}
Q_\mrm{sk} &= \nu
\end{split}
\end{equation}
which denotes the number of times the magnetization winds i.e. winding number of target space on the base space. For the simplest cases, $\nu=1$ implying $Q_\mrm{sk}=1$.
%
\end{document}
