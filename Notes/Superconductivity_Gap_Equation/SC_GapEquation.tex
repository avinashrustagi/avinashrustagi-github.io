\documentclass[aps,prb,onecolumn,notitlepage,showpacs,floatfix,superscriptaddress]{revtex4-1}
\usepackage{dcolumn}
\usepackage{tabularx}
\usepackage{bm}
\usepackage{soul}
\usepackage{amsmath,amssymb,graphicx}
\usepackage[colorlinks=true,citecolor=blue,urlcolor=blue,linkcolor=blue]{hyperref}
\usepackage{environ}

\NewEnviron{eqnsplit}{%
\begin{equation}
\begin{split}
  \BODY
\end{split}
\end{equation}
}

\newcommand{\mrm}[1]{\mathrm{#1}}
\newcommand{\ang}{\mathrm{\AA}}

\bibliographystyle{apsrev4-1}

%%%%%%%%%%%%%%%%%%%%%%%%%%%%%%%%%%%%%%%%%%%%%%%%
\begin{document}

\title{Superconductivity Gap Equation}

\author{Avinash Rustagi}
\email{arustag@ncsu.edu}
\affiliation{Department of Physics, North Carolina State University, Raleigh, NC 27695}
%
\date{\today}
%%%%%%%%%%%%%%%%%%%%%%%%%%%%%%%%%%%%%%%%%%%%%%%%
\begin{abstract}
The objective is to show that the gap equation decouples into different channels.
\end{abstract}

\maketitle
%
The generalized superconductivity gap equation for a given pairing potential $V({\bm k},{\bm p})$,
\begin{equation}
\Delta({\bm k}) = -\sum_{{\bm p}} V({\bm k},{\bm p}) \langle c_{-{\bm p} \downarrow} c_{{\bm p} \uparrow} \rangle
\end{equation}
Superconductivity arises from Fermi surface instability to Cooper pair formation, thus it is fair to restrict the electrons in the Gap equation to Fermi surface (i.e. $k=k_F$ and $p=k_F$). We can express the Gap equation in terms of the anomalous Green function $F({\bm p}) =  \langle c_{-{\bm p} \downarrow} c_{{\bm p} \uparrow} \rangle$
\begin{equation}
\Delta({\bm k}) = -\sum_{{\bm p}} V({\bm k},{\bm p}) F({\bm p})
\end{equation}
Setting $k=k_F$ and $p=k_F$, the gap equation has only angles ($\hat{k}$ and $\hat{p}$)
\begin{equation}
\Delta(\hat{k}) = -\sum_{\hat{p}} V(\hat{k},\hat{p}) F({\hat{ p}}) 
\end{equation}
Assuming azimuthal symmetry, we can expand the functions $\Delta(\hat{k})$ and $F({\hat{ p}})$ in spherical harmonics
\begin{equation}
\begin{split}
\Delta(\hat{k}) &= \sum_{lm} \Delta_{l} Y_{lm}(\hat{k}) \\
F(\hat{p}) &= \sum_{l'm'} F_{l'} Y_{l'm'}(\hat{p}) \\
\end{split}
\end{equation}
We can use the addition theorem of spherical harmonics to expand the pairing potential
\begin{equation}
V({\bm k},{\bm p}) = \sum_{lm} V_{l} Y_{lm}(\hat{k}) Y_{lm}(\hat{p})^*
\end{equation}
Therefore the gap equation in the $l$-channel can be estimated
\begin{equation}
\begin{split}
\Delta_{l} &= \int \dfrac{d\Omega_k}{4\pi} \Delta(\hat{k})  Y_{lm}(\hat{k})^* \\
 &= -\int \dfrac{d\Omega_p}{4\pi} \int \dfrac{d\Omega_k}{4\pi} V(\hat{k},\hat{p}) F({\hat{ p}}) Y_{lm}(\hat{k})^* \\
  &= -\int \dfrac{d\Omega_p}{4\pi} \int \dfrac{d\Omega_k}{4\pi} \sum_{l'm'} V_{l'} Y_{l'm'}(\hat{k}) Y_{l'm'}(\hat{p})^* \sum_{l''m''} F_{l''} Y_{l''m''}(\hat{p})  Y_{lm}(\hat{k})^* \\
\end{split}
\end{equation}
Using the orthogonality of spherical harmonics
\begin{equation}
\int \dfrac{d\Omega_p}{4\pi} Y_{lm}(\hat{p})^* Y_{l'm'}(\hat{p}) = \delta_{l,l'} \delta_{m,m'}
\end{equation}
\begin{equation}
\begin{split}
\Delta_{l} &= - \sum_{l'm'} V_{l'}  \sum_{l''m''} F_{l''} \delta_{l,l'} \delta_{m,m'}\delta_{l',l''} \delta_{m',m''}\\
\end{split}
\end{equation}
which indicates that the different channels decouple
\begin{equation}
\Delta_{l} = - V_{l} \,  F_{l}
\end{equation}
However, we note that we had set all the electron momenta to be at the Fermi surface. In reality, they can have any magnitude within the Debye energy of the Fermi surface,
\begin{equation}
\Delta_{l} = - \int_0^{\hbar \omega_D} d\xi \, N(0) V_{l} \,  F_{l}
\end{equation}
where
\begin{equation}
F_l = \dfrac{\Delta_l}{\sqrt{\Delta_l^2 + \xi^2}} \qquad \xi=\varepsilon-E_F
\end{equation}
\begin{equation}
\Delta_l \approx 2 \hbar \omega_D e^{-1/N(0)V_l}
\end{equation}
\end{document}