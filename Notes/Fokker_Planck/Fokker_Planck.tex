\documentclass[aps,prb,onecolumn,notitlepage,showpacs,floatfix,superscriptaddress]{revtex4-1}
\usepackage{dcolumn}
\usepackage{tabularx}
\usepackage{bm}
\usepackage{soul}
\usepackage{amsmath,amssymb,graphicx}
\usepackage[colorlinks=true,citecolor=blue,urlcolor=blue,linkcolor=blue]{hyperref}
\usepackage{environ}

\NewEnviron{eqnsplit}{%
\begin{equation}
\begin{split}
  \BODY
\end{split}
\end{equation}
}

\newcommand{\mrm}[1]{\mathrm{#1}}
\newcommand{\ang}{\mathrm{\AA}}
\newcommand{\cgr}{C^{>}_{AB}}
\newcommand{\clr}{C^{<}_{AB}}

\bibliographystyle{apsrev4-1}

%%%%%%%%%%%%%%%%%%%%%%%%%%%%%%%%%%%%%%%%%%%%%%%%
\begin{document}
\title{Fokker-Planck Equation}

\author{Avinash Rustagi}
\email{arustag@purdue.edu}
%
\date{\today}
%%%%%%%%%%%%%%%%%%%%%%%%%%%%%%%%%%%%%%%%%%%%%%%%

\maketitle
%
Fokker-Planck equation is a widely used equation that describes the time evolution of the probability of a distribution of Brownian particles that is subject to random forces. Such an equation can be derived in two steps:\\

1) Equation of motion for the probability density $\rho(x,v,t)$ to find the Brownian particle in an interval $(x,x+dx)$ and $(v,v+dv)$ at time $t$ for one realization of the random force $\xi(t)$. \\

2) Average over many realizations of the random force to obtain the macroscopically observed probability density $P(x,v,t) = \langle \rho(x,v,t) \rangle_\xi$. \\

Consider the phase space ($x,v$) and the probability to find the particle in an interval $(x,x+dx)$ and $(v,v+dv)$ at time $t$ is given $\rho(x,v,t) dx dv$. Since the total number of particles is conserved over the entire phase space
\begin{equation}
\int_{-\infty}^{\infty} dx  \int_{-\infty}^{\infty} dv \, \rho(x,v,t) = 1
\end{equation}
Now if we consider the rate of change of particles in volume $V_0$ of the phase space that has a surface $S$, this rate of change is equal to the outflow of particles through the surface $S_0$. Thus by continuity
\begin{equation}
\dfrac{\partial}{\partial t} \int_{V_0} dV \, \rho(x,v,t) = - \int_{S_0} d\vec{S}.\dot{\vec{x}} \, \rho(x,v,t)
\end{equation} 
where $\dot{\vec{x}} = (\dot{x},\dot{v})$ is the velocity in phase space [dot represents the time derivative]. This is simply the ``Continuity Equation" in phase space. By Gauss' Theorem
\begin{equation}
\int_{S_0} d\vec{S}.\dot{\vec{x}} \, \rho(x,v,t) =  \int_{V_0} dV \, \vec{\nabla} \cdot [\dot{\vec{x}} \, \rho(x,v,t)]
\end{equation} 
where $\vec{\nabla} = (\partial_x, \partial_v)$. Therefore
\begin{equation}
\dfrac{\partial}{\partial t}  \rho(x,v,t) = - \dfrac{\partial}{\partial x} [\dot{x} \rho(x,v,t)] - \dfrac{\partial}{\partial v} [\dot{v} \rho(x,v,t)]
\end{equation}
since we can arbitrarily choose the volume $V_0$ in the phase space. \\

For Brownian motion of a particle in a potential $V(x)$ providing force $F(x) = -\partial_x V(x)$,
\begin{equation}
\begin{split}
\dfrac{dx}{dt} &= v \\
\dfrac{dv}{dt} &= - \gamma \dfrac{v}{m}	+ \dfrac{F(x)}{m} + \dfrac{\xi(t)}{m}
\end{split}
\end{equation}
therefore
\begin{equation}
\begin{split}
\dfrac{\partial}{\partial t}  \rho(x,v,t) &= - \dfrac{\partial}{\partial x} [\dot{x} \rho(x,v,t)] - \dfrac{\partial}{\partial v} [\dot{v} \rho(x,v,t)] \\
 &= - \dfrac{\partial}{\partial x} \left[v \rho(x,v,t)\right] + \dfrac{\partial}{\partial v} \left[ \gamma \dfrac{v}{m} \rho(x,v,t)\right] - \dfrac{\partial}{\partial v} \left[ \dfrac{F(x)}{m} \rho(x,v,t)\right] - \dfrac{\partial}{\partial v} \left[ \dfrac{\xi(t)}{m} \rho(x,v,t)\right] \\
 &= -L_0 \rho(x,v,t) - L_1 \rho(x,v,t)
\end{split}
\end{equation}
where the operators $L_0$ and $L_1$ are 
\begin{equation}
\begin{split}
L_0 &= v \dfrac{\partial}{\partial x} - \dfrac{\gamma}{m} - \dfrac{\gamma}{m} v \dfrac{\partial}{\partial v} +  \dfrac{F(x)}{m} \dfrac{\partial}{\partial v} \\
L_1 &= \dfrac{\xi(t)}{m} \dfrac{\partial}{\partial v} \\
\end{split}
\end{equation}
To get to the observable probability density, we need to average over the various realizations of the random force $\xi(t)$
\begin{equation}
P(x,v,t) = \langle \rho(x,v,t)\rangle_\xi
\end{equation}
To evaluate this average, define
\begin{equation}
\rho(x,v,t) = e^{-L_0 t} \sigma(x,v,t)
\end{equation}
which implies
\begin{equation}
\dfrac{\partial}{\partial t}  \sigma(x,v,t) = - e^{L_0 t} L_1 e^{-L_0 t} \sigma(x,v,t) \equiv - V(t) \sigma(x,v,t)
\end{equation}
The formal solution to this equation is 
\begin{equation}
\sigma(x,v,t) = \exp\left[-\int_0^t dt_1 \, V(t_1) \right] \sigma(x,v,0)
\end{equation}
Averaging over the random force realizations
\begin{equation}
\langle \sigma(x,v,t) \rangle_\xi = \langle \exp\left[-\int_0^t dt_1 \, V(t_1) \right] \rangle_\xi  \sigma(x,v,0)
\end{equation}
which upon using the cumulant expansion relation 
\begin{equation}
\langle e^{-i \Phi(t)} \rangle = \exp \left[ \sum_{n=1}^{\infty} \dfrac{(-i )^n}{n!} c_n \right]
\end{equation}
gives (assuming that the random force is Gaussian implying that only second cumulant is non-zero equivalent to stating that only even moments are non-zero)
\begin{equation}
\langle \sigma(x,v,t) \rangle_\xi = \exp\left[\dfrac{1}{2}\int_0^t dt_1 \, \int_0^t dt_2 \, \langle V(t_1) V(t_2)\rangle_\xi \right]   \sigma(x,v,0)
\end{equation}
Thus we evaluate the average in the exponential
\begin{equation}
\begin{split}
\dfrac{1}{2}\int_0^t dt_1 \, \int_0^t dt_2 \, \langle V(t_1) V(t_2)\rangle_\xi &= \dfrac{1}{2}\int_0^t dt_1 \, \int_0^t dt_2 \, \langle e^{L_0 t_1} \dfrac{\xi(t_1)}{m} \dfrac{\partial}{\partial v} e^{-L_0 t_1} e^{L_0 t_2} \dfrac{\xi(t_2)}{m} \dfrac{\partial}{\partial v} e^{-L_0 t_2}\rangle_\xi \\
&= \dfrac{1}{2}\int_0^t dt_1 \, \int_0^t dt_2 \, \dfrac{\langle \xi(t_1) \xi(t_2)\rangle_\xi}{m^2}   e^{L_0 t_1}  \dfrac{\partial}{\partial v} e^{-L_0 t_1} e^{L_0 t_2} \dfrac{\partial}{\partial v} e^{-L_0 t_2} \\
&= \dfrac{1}{2}\int_0^t dt_1 \, \int_0^t dt_2 \, \dfrac{g \delta(t_1-t_2)}{m^2}   e^{L_0 t_1}  \dfrac{\partial}{\partial v} e^{-L_0 t_1} e^{L_0 t_2} \dfrac{\partial}{\partial v} e^{-L_0 t_2} \\
&= \dfrac{1}{2}\int_0^t dt_1 \, \dfrac{g}{m^2}   e^{L_0 t_1}  \dfrac{\partial^2}{\partial v^2} e^{-L_0 t_1} \\
\end{split}
\end{equation}
where we have used the Gaussian nature of the random force i.e. $\langle \xi(t_1) \xi(t_2)\rangle_\xi = g \delta(t_1-t_2)$. Thus
\begin{equation}
\langle \sigma(x,v,t) \rangle_\xi = \exp\left[\dfrac{g}{2m^2}\int_0^t dt_1 \,  e^{L_0 t_1}  \dfrac{\partial^2}{\partial v^2} e^{-L_0 t_1}\right]   \sigma(x,v,0)
\end{equation}
Taking the time-derivative of the above equation
\begin{equation}
\dfrac{\partial}{\partial t}\langle \sigma(x,v,t) \rangle_\xi = \dfrac{g}{2m^2}  e^{L_0 t}  \dfrac{\partial^2}{\partial v^2} e^{-L_0 t}  \langle \sigma(x,v,t) \rangle_\xi
\end{equation}
which translates to
\begin{equation}
\dfrac{\partial}{\partial t}\langle \rho(x,v,t) \rangle_\xi = -L_0 \langle \rho(x,v,t) \rangle_\xi + \dfrac{g}{2m^2}  \dfrac{\partial^2}{\partial v^2}  \langle \rho(x,v,t) \rangle_\xi
\end{equation}
which is the Fokker-Planck equation in terms of the macroscopic probability density
\begin{equation}
\dfrac{\partial}{\partial t} P(x,v,t) = - v \dfrac{\partial}{\partial x}  P(x,v,t) + \dfrac{\partial}{\partial v} \left[ \left(\dfrac{\gamma}{m} v  -  \dfrac{F(x)}{m} \right)  P(x,v,t) \right] + \dfrac{g}{2m^2}  \dfrac{\partial^2}{\partial v^2}  P(x,v,t) 
\end{equation}
%
In absence of an external force and in thermal equilibrium ($\partial_t P = 0$), the probability distribution is given by the Boltzmann factor
\begin{equation}
P_0 \propto e^{-\beta m v^2/2} \Rightarrow \partial_v P_0 = -\beta m v P_0
\end{equation}
Hence
\begin{equation}
\begin{split}
0 &= \dfrac{\partial}{\partial v} \left[ \dfrac{\gamma}{m} v   P_0(v) \right] + \dfrac{g}{2m^2}  \dfrac{\partial^2}{\partial v^2} P_0(v) \\
0 &= \dfrac{\partial}{\partial v} \left[ \dfrac{\gamma}{m} v P_0(v) \right] - \dfrac{g}{2m^2}  \dfrac{\partial}{\partial v} \beta m v P_0 \\
0 &= \dfrac{\partial}{\partial v} \left[ \left(\dfrac{\gamma}{m} - \dfrac{\beta g}{2m}\right)  v P_0(v) \right]
\end{split}
\end{equation}
which implies 
\begin{equation}
g=2 \gamma k_B T 
\end{equation}
\section*{Mathematical Relations}
\subsection*{Moments and Characteristic Function}
Probability distribution functions (PDF) are normalized:
\begin{equation}
\int_{-\infty}^{\infty} dx \, P(x) = 1
\end{equation}
which implies that the Fourier component of PDF at $k=0$ is unity. The Fourier transform of the PDF can be defined as
\begin{equation}
P(k) = \int_{-\infty}^{\infty} dx \, e^{-ikx} \, P(x) 
\end{equation} 
and from the normalization condition $P(k=0)=1$. The function $P(k)$ is referred to as the ``Characteristic Function". The moments of the PDF can be thereby expressed in terms of the derivatives of the Characteristic Function.
\begin{equation}
\begin{split}
m_1 &= \langle x \rangle = \int_{-\infty}^{\infty} dx \, x \, P(x) = i \dfrac{\partial P(k)}{\partial k} \bigg\vert_{k=0} \\
m_2 &= \langle x^2 \rangle = \int_{-\infty}^{\infty} dx \, x^2 \, P(x) = i^2 \dfrac{\partial^2 P(k)}{\partial k^2} \bigg\vert_{k=0} \\
\vdots \\
m_n &= \langle x^n \rangle = \int_{-\infty}^{\infty} dx \, x^n \, P(x) = i^n \dfrac{\partial^n P(k)}{\partial k^n} \bigg\vert_{k=0} \\
\end{split}
\end{equation}
Therefore
\begin{equation}
P(k) = \sum_{n=0}^{\infty} \dfrac{(-i k)^n}{n!} m_n
\end{equation}

\subsection*{Cumulants and Cumulant Generating Function}
From the relation between the PDF and the characteristic function
\begin{equation}
P(x) = \int_{-\infty}^{\infty} \dfrac{dk}{2\pi} \, e^{ikx} \, P(k) 
\end{equation} 
a ``Cumulant Generating Function" $\psi(k)$ is defined as 
\begin{equation}
P(x) = \int_{-\infty}^{\infty} \dfrac{dk}{2\pi} \, e^{ikx} \, e^{\psi(k)}
\end{equation} 
where $\psi(k) = \mrm{Log}[P(k)]$ is the function whose Taylor series coefficients at the origin $k=0$ are the ``Cumulants".
\begin{equation}
c_n = \dfrac{1}{i^n} \dfrac{\partial^n \psi(k)}{\partial k^n} \bigg\vert_{k=0} 
\end{equation}
Therefore
\begin{equation}
\begin{split}
\psi(k) &= -i k c_1 -\dfrac{1}{2!} k^2 c_2 .... \\
&= \sum_{n=1}^{\infty} \dfrac{(-i k)^n}{n!} c_n
\end{split}
\end{equation}
Comparing to the Characteristic function expansion in terms of moments
\begin{equation}
\psi(k) =  \sum_{n=1}^{\infty} \dfrac{(-i k)^n}{n!} c_n = \mrm{Log} \left[  \sum_{n=0}^{\infty} \dfrac{(-i k)^n}{n!} m_n \right]
\end{equation}
implies
\begin{itemize}
\item $c_1 = m_1$ which is the ``Mean"
\item $c_2 = m_2 - m_1^2 = \sigma^2$ which is the ``Variance" [$\sigma$: Standard Deviation]
\item $c_3 = m_3 -3 m_1 m_2 +2 m_1^3$ which is the ``Skewness"
\item $c_4 = m_4 -3 m_2^2 -4 m_1 m_3 + 12 m_1^2 m_2 -6 m_1^4$ which is the ``Kurtosis"
\end{itemize}
Therefore
\begin{equation}
\begin{split}
P(k) = \exp \left[ \sum_{n=1}^{\infty} \dfrac{(-i k)^n}{n!} c_n \right] = \sum_{n=0}^{\infty} \dfrac{(-i k)^n}{n!} m_n 
\end{split}
\end{equation}
which implies
\begin{equation}
\begin{split}
P(k=1) = \exp \left[ \sum_{n=1}^{\infty} \dfrac{(-i )^n}{n!} c_n \right] = \sum_{n=0}^{\infty} \dfrac{(-i )^n}{n!} m_n 
\end{split}
\end{equation}
Consider the following average
\begin{equation}
\begin{split}
\langle e^{-i \Phi(t)} \rangle &= \langle \sum_{n=0}^{\infty} \dfrac{(-i )^n}{n!} \Phi(t)^n \rangle \\
&=  \sum_{n=0}^{\infty} \dfrac{(-i )^n}{n!}  \langle\Phi(t)^n \rangle \\
&=  \sum_{n=0}^{\infty} \dfrac{(-i )^n}{n!} m_n \\
&= \exp \left[ \sum_{n=1}^{\infty} \dfrac{(-i )^n}{n!} c_n \right]
\end{split}
\end{equation}
\end{document}
