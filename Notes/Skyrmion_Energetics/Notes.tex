\documentclass[aps,prb,onecolumn,notitlepage,showpacs,floatfix,superscriptaddress]{revtex4-1}
\usepackage{dcolumn}
\usepackage{tabularx}
\usepackage{bm}
\usepackage{soul}
\usepackage{amsmath,xcolor}
\fboxrule=1pt
\usepackage{amssymb,graphicx}
\usepackage[colorlinks=true,citecolor=blue,urlcolor=blue,linkcolor=blue]{hyperref}
\usepackage{environ}

\usepackage{tikz}
\usetikzlibrary{matrix}
\usetikzlibrary{fit}

\NewEnviron{eqnsplit}{%
\begin{equation}
\begin{split}
  \BODY
\end{split}
\end{equation}
}
\newcommand{\mrm}[1]{\mathrm{#1}}
\newcommand{\AR}[1]{\textcolor{red}{#1}}
\newcommand{\ang}{\mathrm{\AA}}

\bibliographystyle{apsrev4-1}

%%%%%%%%%%%%%%%%%%%%%%%%%%%%%%%%%%%%%%%%%%%%%%%%
\begin{document}
\title{Neel Skyrmion: Energetics}

\author{Avinash Rustagi}
\email{arustag@purdue.edu}
%
%\date{\today}
%%%%%%%%%%%%%%%%%%%%%%%%%%%%%%%%%%%%%%%%%%%%%%%%

\maketitle
%
Let us consider a exchange coupled magnetic thin film (thickness $L$) with an out-of-plane uniaxial easy axis, interfacial DMI (caused by a heavy metal substrate), and an externally applied magnetic field along the out-of-plane normal. The energy describing this magnet is
\begin{equation}
\begin{split}
E_\mrm{total} &= E_\mrm{ex} + E_\mrm{uni} + E_\mrm{dmi} + E_\mrm{z}\\
\end{split}
\end{equation}
The magnetization unit vector can be described in terms of angles $\Theta (\vec{r})$ and $\Phi(\vec{r})=\nu \phi + \gamma$ which are functions of in-plane spatial coordinates ($\vec{r}=\{r\cos\phi, r\sin\phi\}$) such that
\begin{equation}
\begin{split}
E_\mrm{z} &= M_s B L \int dS\, [1-m_z] \\
&= 2\pi M_s B L \int rdr\, [1-\cos\Theta]\\
E_\mrm{uni} &= M_s K_u L \int dS\, [1-m_z^2] \\
&= 2\pi M_s K_u L \int rdr\, \sin^2\Theta\\ 
E_\mrm{ex} &= A L \int dS\, [(\vec{\nabla} m_x)^2+(\vec{\nabla} m_y)^2+(\vec{\nabla} m_z)^2] \\
&= 2\pi A L \int rdr\, \left[ \left(\dfrac{d\Theta}{dr}\right)^2 + \dfrac{\sin^2\Theta}{r^2}\right]\\ 
E_\mrm{dmi} &= D_\mrm{dmi} L \int dS\, [m_z \vec{\nabla} \cdot \vec{m}- \vec{m}\cdot \vec{\nabla} m_z] \\
&= 2\pi D_\mrm{dmi} L \cos\gamma \int rdr\, \left[ \dfrac{d\Theta}{dr} + \dfrac{\sin2\Theta}{2r}\right]\\ 
\end{split}
\end{equation}
where $\nu$ is the vorticity (winding number; set to be 1) and $\gamma$ defines the skyrmion type ($\gamma=0/\pi$- Neel; $\gamma=\pm\pi/2$- Bloch). Defining polarity as
\begin{equation}
p = \dfrac{m_z(0)-m_{z}(\infty)}{2}
\end{equation}
and considering the situation 
\begin{itemize}
\item where $p=1$ meaning magnetization is along +z at the skyrmion core center and along -z outside the skyrmion core, we can see that the angle $\Theta$ increases monotonically as one moves radially outward from the core center implying $d\Theta/dr >0$. On the other hand the integral of $\sin\Theta \cos\Theta$ is very small as the product has opposite signs on either side of the Skyrmion wall. Assuming that the interfacial DMI energy $D_\mrm{dmi}>0$, it can be seen that $\gamma=\pi$ will lower the energy of the system. Thus, we would have a Neel Skyrmion with $\gamma=\pi$.
\item where $p=-1$ meaning magnetization is along -z at the skyrmion core center and along +z outside the skyrmion core, we can see that the angle $\Theta$ decreases monotonically as one moves radially outward from the core center implying $d\Theta/dr < 0$. On the other hand the integral of $\sin\Theta \cos\Theta$ is very small as the product has opposite signs on either side of the Skyrmion wall. Assuming that the interfacial DMI energy $D_\mrm{dmi}>0$, it can be seen that $\gamma=0$ will lower the energy of the system. Thus, we would have a Neel Skyrmion with $\gamma=0$.
\end{itemize}


\noindent An ansatz for the Neel Skyrmion (of radius $R_s$ and wall width $w_s$) with polarity 1 is
\begin{equation}
m_z = \dfrac{\sinh(R_s/w_s)-\sinh(r/w_s)}{\sinh(R_s/w_s)+\sinh(r/w_s)}
\end{equation}
equivalent to 
\begin{equation}
\Theta = 2 \tan^{-1} \left[ \dfrac{\sinh(R_s/w_s)}{\sinh(r/w_s)}\right]
\end{equation}
Following which
\begin{equation}
\begin{split}
m_x &= \sqrt{1-m_z^2} \cos[\nu \phi + \gamma] \\
m_y &= \sqrt{1-m_z^2} \sin[\nu \phi + \gamma] \\
\end{split}
\end{equation}
Note that the ansatz can be simply generalized for Bloch skyrmions (However, we might need free energy terms corresponding to Bulk DMI and dipole-dipole interaction for understanding the stability of the soliton).
%
\end{document}
