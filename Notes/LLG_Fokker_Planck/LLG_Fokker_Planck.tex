\documentclass[aps,prb,onecolumn,notitlepage,showpacs,floatfix,superscriptaddress]{revtex4-1}
\usepackage{dcolumn}
\usepackage{tabularx}
\usepackage{bm}
\usepackage{soul}
\usepackage{amsmath,amssymb,graphicx}
\usepackage[colorlinks=true,citecolor=blue,urlcolor=blue,linkcolor=blue]{hyperref}
\usepackage{environ}

\NewEnviron{eqnsplit}{%
\begin{equation}
\begin{split}
  \BODY
\end{split}
\end{equation}
}

\newcommand{\mrm}[1]{\mathrm{#1}}
\newcommand{\ang}{\mathrm{\AA}}
\newcommand{\cgr}{C^{>}_{AB}}
\newcommand{\clr}{C^{<}_{AB}}

\bibliographystyle{apsrev4-1}

%%%%%%%%%%%%%%%%%%%%%%%%%%%%%%%%%%%%%%%%%%%%%%%%
\begin{document}
\title{Landau-Lifshitz-Gilbert - Fokker-Planck Equation}

\author{Avinash Rustagi}
\email{arustag@purdue.edu}
%
\date{\today}
%%%%%%%%%%%%%%%%%%%%%%%%%%%%%%%%%%%%%%%%%%%%%%%%

\maketitle
%
Landau-Lifshitz-Gilbert equation (equivalent to the Landau-Lifshitz equation with renormalized parameters) governs the magnetization dynamics in an effective magnetic field $\vec{B}$ determined by the free energy landscape. It is simple to generalize it to include a fluctuating field $\vec{b}(t)$.
\begin{equation}
\begin{split}
\dfrac{d\vec{M}}{dt} &= -\gamma_G \vec{M} \times (\vec{B}+\vec{b}(t)) +\dfrac{\alpha }{M_s} \vec{M} \times \dfrac{d\vec{M}}{dt} \\
&= -\gamma_G \vec{M} \times (\vec{B}+\vec{b}(t)) +\dfrac{\alpha }{M_s} \vec{M} \times \left[ -\gamma_G \vec{M} \times (\vec{B}+\vec{b}(t)) +\dfrac{\alpha }{M_s} \vec{M} \times \dfrac{d\vec{M}}{dt}\right] \\
&= -\gamma_G \vec{M} \times (\vec{B}+\vec{b}(t)) -\gamma_G \dfrac{\alpha }{M_s} \vec{M} \times \vec{M} \times (\vec{B}+\vec{b}(t)) +\dfrac{\alpha^2 }{M_s^2} \vec{M} \times \vec{M} \times \dfrac{d\vec{M}}{dt} \\
&= -\gamma_G \vec{M} \times (\vec{B}+\vec{b}(t)) -\gamma_G \dfrac{\alpha }{M_s} \vec{M} \times \vec{M} \times (\vec{B}+\vec{b}(t)) -\alpha^2 \dfrac{d\vec{M}}{dt} \\
&= -\dfrac{\gamma_G}{1+\alpha^2} \vec{M} \times (\vec{B}+\vec{b}(t)) - \dfrac{\alpha \gamma_G }{M_s (1+\alpha^2)} \vec{M} \times \vec{M} \times (\vec{B}+\vec{b}(t)) \\
\end{split}
\end{equation}
Defining $\gamma = \gamma_G /(1+\alpha^2)$:
\begin{equation}
\dfrac{d\vec{M}}{dt} = -\gamma \vec{M} \times (\vec{B}+\vec{b}(t)) -\dfrac{\alpha \gamma}{M_s} \vec{M} \times \vec{M} \times (\vec{B}+\vec{b}(t))
\end{equation}
where the fluctuating field $\vec{b}(t)$ is a Gaussian stochastic variable
\begin{equation}
\langle b_i(t) \rangle = 0 \qquad \langle b_i(t_1) b_j(t_2) \rangle = 2 D \delta_{ij} \delta(t_1-t_2)
\end{equation}
Assuming that damping is weak, we can drop the fluctuating field from the second term in the LLG equation
\begin{equation}
\dfrac{d\vec{M}}{dt} = -\gamma \vec{M} \times (\vec{B}+\vec{b}(t)) -\dfrac{\alpha \gamma}{M_s} \vec{M} \times \vec{M} \times \vec{B}
\end{equation}
Considering the Fokker-Planck equation for the probability density of having magnetization $\vec{M}$ at time $t$ for a given realization of stochastic field is
\begin{equation}
\begin{split}
\dfrac{\partial}{\partial t}  \rho(\vec{M},t) &= - \dfrac{\partial}{\partial \vec{M}} \cdot [\dot{\vec{M}} \rho(\vec{M},t)] \\
 &= - \dfrac{\partial}{\partial \vec{M}} \cdot \left[\left(-\gamma \vec{M} \times (\vec{B}+\vec{b}(t)) -\dfrac{\alpha \gamma}{M_s} \vec{M} \times \vec{M} \times \vec{B}\right) \rho(\vec{M},t) \right]  \\
 &=  \dfrac{\partial}{\partial \vec{M}} \cdot \left[\left(\gamma \vec{M} \times \vec{B} +\dfrac{\alpha \gamma}{M_s} \vec{M} \times \vec{M} \times \vec{B}\right) \rho(\vec{M},t) \right] +  \dfrac{\partial}{\partial \vec{M}} \cdot \left[\gamma \vec{M} \times \vec{b}(t)\rho(\vec{M},t) \right]\\
 &=  \partial_{M_i} \left[\left(\gamma \epsilon_{ijk}M_j B_k +\dfrac{\alpha \gamma}{M_s} \epsilon_{ijk}M_j \epsilon_{kpq}M_p B_q \right) \rho(\vec{M},t) \right] +  \partial_{M_i} \left[\gamma \epsilon_{ijk} M_j b_k(t) \rho(\vec{M},t) \right]\\
 &=  \left[\gamma \epsilon_{ijk}M_j B_k \partial_{M_i} +\dfrac{\alpha \gamma}{M_s} \epsilon_{ijk} \epsilon_{kpq} M_j \partial_{M_i} M_p B_q \right]  \rho(\vec{M},t) +   \left[\gamma \epsilon_{ijk} M_j b_k(t) \partial_{M_i}\right] \rho(\vec{M},t) \\
 &= -L_0  \rho(\vec{M},t) - L_1  \rho(\vec{M},t)
\end{split}
\end{equation}
where the operators $L_0$ and $L_1$ are 
\begin{equation}
\begin{split}
L_0 &= -\left[\gamma \epsilon_{ijk}M_j B_k \partial_{M_i} +\dfrac{\alpha \gamma}{M_s} \epsilon_{ijk} \epsilon_{kpq} M_j \partial_{M_i} M_p B_q \right] \\
L_1 &=-\left[\gamma \epsilon_{ijk} M_j b_k(t) \partial_{M_i}\right]\\
\end{split}
\end{equation}
To get to the observable probability density, we need to average over the various realizations of the random force $\xi(t)$
\begin{equation}
P(\vec{M},t) = \langle \rho(\vec{M},t)\rangle_b
\end{equation}
To evaluate this average, define
\begin{equation}
\rho(\vec{M},t) = e^{-L_0 t} \sigma(\vec{M},t)
\end{equation}
which implies
\begin{equation}
\dfrac{\partial}{\partial t}  \sigma(\vec{M},t) = - e^{L_0 t} L_1 e^{-L_0 t} \sigma(\vec{M},t) \equiv - V(t) \sigma(\vec{M},t)
\end{equation}
The formal solution to this equation is 
\begin{equation}
\sigma(\vec{M},t) = \exp\left[-\int_0^t dt_1 \, V(t_1) \right] \sigma(\vec{M},t=0)
\end{equation}
Averaging over the random force realizations
\begin{equation}
\langle \sigma(\vec{M},t) \rangle_\xi = \langle \exp\left[-\int_0^t dt_1 \, V(t_1) \right] \rangle_b  \sigma(\vec{M},t=0)
\end{equation}
which upon using the cumulant expansion relation 
\begin{equation}
\langle e^{-i \Phi(t)} \rangle = \exp \left[ \sum_{n=1}^{\infty} \dfrac{(-i )^n}{n!} c_n \right]
\end{equation}
gives (assuming that the random force is Gaussian implying that only second cumulant is non-zero equivalent to stating that only even moments are non-zero)
\begin{equation}
\langle \sigma(\vec{M},t) \rangle_b = \exp\left[\dfrac{1}{2}\int_0^t dt_1 \, \int_0^t dt_2 \, \langle V(t_1) V(t_2)\rangle_b \right]   \sigma(\vec{M},t=0)
\end{equation}
Thus we evaluate the average in the exponential
\begin{equation}
\begin{split}
\dfrac{1}{2}\int_0^t dt_1 \, \int_0^t dt_2 \, \langle V(t_1) V(t_2)\rangle_b &= \dfrac{1}{2}\int_0^t dt_1 \, \int_0^t dt_2 \, \langle e^{L_0 t_1}  \gamma \epsilon_{ijk} M_j b_k(t_1) \partial_{M_i} e^{-L_0 t_1} e^{L_0 t_2} \gamma \epsilon_{pqr} M_q b_r(t_2) \partial_{M_p} e^{-L_0 t_2}\rangle_b \\
&= \dfrac{1}{2}\int_0^t dt_1 \, \int_0^t dt_2 \, \gamma^2 \langle b_k(t_1) b_r(t_2) \rangle_b e^{L_0 t_1}   \epsilon_{ijk} M_j  \partial_{M_i} e^{-L_0 t_1} e^{L_0 t_2} \epsilon_{pqr}  M_q \partial_{M_p} e^{-L_0 t_2} \\
&= \dfrac{1}{2}\int_0^t dt_1 \, \int_0^t dt_2 \, \gamma^2 2 D \delta_{kr} \delta(t_1-t_2) e^{L_0 t_1}   \epsilon_{ijk} M_j  \partial_{M_i} e^{-L_0 t_1} e^{L_0 t_2}  \epsilon_{pqr}  M_q \partial_{M_p} e^{-L_0 t_2} \\
&= D \gamma^2 \int_0^t dt_1 \, e^{L_0 t_1}   \epsilon_{ijk} M_j  \partial_{M_i} \epsilon_{pqk}  M_q \partial_{M_p} e^{-L_0 t_1} \\
\end{split}
\end{equation}
where we have used the Gaussian nature of the random field i.e. $\langle b_k(t_1) b_r(t_2) \rangle = 2 D \delta_{kr} \delta(t_1-t_2)$. Thus
\begin{equation}
\langle \sigma(\vec{M},t) \rangle_b = \exp\left[D \gamma^2 \int_0^t dt_1 \, e^{L_0 t_1}   \epsilon_{ijk} M_j  \partial_{M_i} \epsilon_{pqk}  M_q \partial_{M_p} e^{-L_0 t_1} \right]   \sigma(\vec{M},0)
\end{equation}
Taking the time-derivative of the above equation
\begin{equation}
\dfrac{\partial}{\partial t}\langle \sigma(\vec{M},t) \rangle_b = D \gamma^2 e^{L_0 t}  \epsilon_{ijk} M_j  \partial_{M_i} \epsilon_{pqk}  M_q \partial_{M_p} e^{-L_0 t}  \langle \sigma(\vec{M},t) \rangle_b
\end{equation}
which translates to
\begin{equation}
\dfrac{\partial}{\partial t}\langle \rho(\vec{M},t) \rangle_b = -L_0 \langle \rho(\vec{M},t) \rangle_b +D \gamma^2  \epsilon_{ijk} M_j  \partial_{M_i} \epsilon_{pqk}  M_q \partial_{M_p} \langle \rho(\vec{M},t) \rangle_b
\end{equation}
which is the Fokker-Planck equation in terms of the macroscopic probability density
\begin{equation}
\dfrac{\partial}{\partial t} P(\vec{M},t) = -L_0 P(\vec{M},t)  +D \gamma^2  \epsilon_{ijk} M_j  \partial_{M_i} \epsilon_{pqk}  M_q \partial_{M_p} P(\vec{M},t) 
\end{equation}
which can be simplified as
\begin{equation}
\epsilon_{ijk} M_j  \partial_{M_i} \epsilon_{pqk}  M_q \partial_{M_p} =  \partial_{M_i} \epsilon_{ijk} M_j   \epsilon_{kpq}  M_q \partial_{M_p} =  -\partial_{M_i} \epsilon_{ijk} M_j   \epsilon_{kqp}  M_q \partial_{M_p} \equiv -\dfrac{\partial}{\partial \vec{M}} \cdot \left[\vec{M} \times \vec{M} \times \dfrac{\partial}{\partial \vec{M}}   \right]
\end{equation}
and 
\begin{equation}
\begin{split}
- L_0 P(\vec{M},t)  &=\dfrac{\partial}{\partial \vec{M}} \cdot \left[\left(\gamma \vec{M} \times \vec{B} +\dfrac{\alpha \gamma}{M_s} \vec{M} \times \vec{M} \times \vec{B}\right) P(\vec{M},t) \right]  \\
\end{split}
\end{equation}
Therefore, the final form of the Fokker Planck equation is
\begin{equation}
\begin{split}
\dfrac{\partial}{\partial t} P(\vec{M},t) &= \dfrac{\partial}{\partial \vec{M}} \cdot \left[\left(\gamma \vec{M} \times \vec{B} +\dfrac{\alpha \gamma}{M_s} \vec{M} \times \vec{M} \times \vec{B}\right) P(\vec{M},t) \right]  - D \gamma^2 \dfrac{\partial}{\partial \vec{M}} \cdot \left[\left(\vec{M} \times \vec{M} \times \dfrac{\partial}{\partial \vec{M}} \right) P(\vec{M},t)  \right] \\
 &= \dfrac{\partial}{\partial \vec{M}} \cdot \left[\left(\gamma \vec{M} \times \vec{B} +\dfrac{\alpha \gamma}{M_s} \vec{M} \times \vec{M} \times \vec{B}\right) P(\vec{M},t) - D \gamma^2 \left(\vec{M} \times \vec{M} \times \dfrac{\partial}{\partial \vec{M}} \right) P(\vec{M},t) \right]  
 \end{split}
\end{equation}
%
\subsection*{Thermal Equilibrium}
In absence of an external force and in thermal equilibrium ($\partial_t P = 0$), the probability distribution is given by the Boltzmann factor
\begin{equation}
P_0 \propto e^{-\beta \mathcal{H}} \Rightarrow \dfrac{\partial}{\partial \vec{M}} P_0 = \beta \vec{B} P_0
\end{equation}
where $\mathcal{H}$ is the free energy and thus the effective field is 
\begin{equation}
\vec{B} = - \dfrac{\partial \mathcal{H}}{\partial \vec{M}}
\end{equation}
This means that the quantity $\gamma (\vec{M} \times \vec{B}) P_0(\vec{M})$ is divergence-less i.e.
\begin{equation}
\begin{split}
\dfrac{\partial}{\partial \vec{M}} \cdot\left[ (\vec{M} \times \vec{B}) P_0(\vec{M}) \right] &=  \left[ \dfrac{\partial}{\partial \vec{M}} \cdot(\vec{M} \times \vec{B}) \right] P_0(\vec{M}) + (\vec{M} \times \vec{B}) \cdot \dfrac{\partial}{\partial \vec{M}} P_0 (\vec{M}) \\
&=  \left[ \dfrac{\partial}{\partial \vec{M}} \cdot(\vec{M} \times \vec{B}) \right] P_0(\vec{M}) + \beta (\vec{M} \times \vec{B}) \cdot  \vec{B} P_0 \\
&=0
\end{split}
\end{equation}
Hence, from the Fokker-Planck equation
\begin{equation}
\begin{split}
0 &= \dfrac{\partial}{\partial \vec{M}} \cdot \left[\left( \dfrac{\alpha \gamma}{M_s} \vec{M} \times \vec{M} \times \vec{B}\right) P_0(\vec{M}) - D \gamma^2 \left(\vec{M} \times \vec{M} \times \dfrac{\partial}{\partial \vec{M}} \right) P_0(\vec{M}) \right]  \\
&= \dfrac{\partial}{\partial \vec{M}} \cdot \left[\left( \dfrac{\alpha \gamma}{M_s} \vec{M} \times \vec{M} \times \vec{B}\right) P_0(\vec{M}) - \beta D \gamma^2 \left(\vec{M} \times \vec{M} \times \vec{B} \right) P_0(\vec{M}) \right]  
 \end{split}
\end{equation}
which implies 
\begin{equation}
D=\dfrac{\alpha k_B T }{\gamma M_s}
\end{equation}
Thus
\begin{equation}
\langle b_i(t_1) b_j(t_2) \rangle = \dfrac{2 \alpha k_B T }{\gamma M_s} \delta_{ij} \delta(t_1-t_2)
\end{equation}
\subsection*{Diffusion Timescale}
To consider the timescale corresponding to pure diffusion, we can set the external effective field to zero $\vec{B}=0$. Therefore, the Fokker Planck equation takes the simplified form
\begin{equation}
\begin{split}
\dfrac{\partial}{\partial t} P(\vec{M},t) &= - D \gamma^2 \dfrac{\partial}{\partial \vec{M}} \cdot \left[\left(\vec{M} \times \vec{M} \times \dfrac{\partial}{\partial \vec{M}} \right) P(\vec{M},t) \right]  
 \end{split}
\end{equation}
It is clear that the only timescale in this pure diffusion process happens to be
\begin{equation}
t_D^{-1} = D \gamma^2 = \dfrac{\alpha k_B T }{\gamma M_s} \gamma^2 = \dfrac{\alpha \gamma k_B T }{ M_s}  = \dfrac{\alpha \gamma_G k_B T }{ M_s (1+\alpha^2)} 
\end{equation}
\section*{Mathematical Relations}
\subsection*{Moments and Characteristic Function}
Probability distribution functions (PDF) are normalized:
\begin{equation}
\int_{-\infty}^{\infty} dx \, P(x) = 1
\end{equation}
which implies that the Fourier component of PDF at $k=0$ is unity. The Fourier transform of the PDF can be defined as
\begin{equation}
P(k) = \int_{-\infty}^{\infty} dx \, e^{-ikx} \, P(x) 
\end{equation} 
and from the normalization condition $P(k=0)=1$. The function $P(k)$ is referred to as the ``Characteristic Function". The moments of the PDF can be thereby expressed in terms of the derivatives of the Characteristic Function.
\begin{equation}
\begin{split}
m_1 &= \langle x \rangle = \int_{-\infty}^{\infty} dx \, x \, P(x) = i \dfrac{\partial P(k)}{\partial k} \bigg\vert_{k=0} \\
m_2 &= \langle x^2 \rangle = \int_{-\infty}^{\infty} dx \, x^2 \, P(x) = i^2 \dfrac{\partial^2 P(k)}{\partial k^2} \bigg\vert_{k=0} \\
\vdots \\
m_n &= \langle x^n \rangle = \int_{-\infty}^{\infty} dx \, x^n \, P(x) = i^n \dfrac{\partial^n P(k)}{\partial k^n} \bigg\vert_{k=0} \\
\end{split}
\end{equation}
Therefore
\begin{equation}
P(k) = \sum_{n=0}^{\infty} \dfrac{(-i k)^n}{n!} m_n
\end{equation}

\subsection*{Cumulants and Cumulant Generating Function}
From the relation between the PDF and the characteristic function
\begin{equation}
P(x) = \int_{-\infty}^{\infty} \dfrac{dk}{2\pi} \, e^{ikx} \, P(k) 
\end{equation} 
a ``Cumulant Generating Function" $\psi(k)$ is defined as 
\begin{equation}
P(x) = \int_{-\infty}^{\infty} \dfrac{dk}{2\pi} \, e^{ikx} \, e^{\psi(k)}
\end{equation} 
where $\psi(k) = \mrm{Log}[P(k)]$ is the function whose Taylor series coefficients at the origin $k=0$ are the ``Cumulants".
\begin{equation}
c_n = \dfrac{1}{i^n} \dfrac{\partial^n \psi(k)}{\partial k^n} \bigg\vert_{k=0} 
\end{equation}
Therefore
\begin{equation}
\begin{split}
\psi(k) &= -i k c_1 -\dfrac{1}{2!} k^2 c_2 .... \\
&= \sum_{n=1}^{\infty} \dfrac{(-i k)^n}{n!} c_n
\end{split}
\end{equation}
Comparing to the Characteristic function expansion in terms of moments
\begin{equation}
\psi(k) =  \sum_{n=1}^{\infty} \dfrac{(-i k)^n}{n!} c_n = \mrm{Log} \left[  \sum_{n=0}^{\infty} \dfrac{(-i k)^n}{n!} m_n \right]
\end{equation}
implies
\begin{itemize}
\item $c_1 = m_1$ which is the ``Mean"
\item $c_2 = m_2 - m_1^2 = \sigma^2$ which is the ``Variance" [$\sigma$: Standard Deviation]
\item $c_3 = m_3 -3 m_1 m_2 +2 m_1^3$ which is the ``Skewness"
\item $c_4 = m_4 -3 m_2^2 -4 m_1 m_3 + 12 m_1^2 m_2 -6 m_1^4$ which is the ``Kurtosis"
\end{itemize}
Therefore
\begin{equation}
\begin{split}
P(k) = \exp \left[ \sum_{n=1}^{\infty} \dfrac{(-i k)^n}{n!} c_n \right] = \sum_{n=0}^{\infty} \dfrac{(-i k)^n}{n!} m_n 
\end{split}
\end{equation}
which implies
\begin{equation}
\begin{split}
P(k=1) = \exp \left[ \sum_{n=1}^{\infty} \dfrac{(-i )^n}{n!} c_n \right] = \sum_{n=0}^{\infty} \dfrac{(-i )^n}{n!} m_n 
\end{split}
\end{equation}
Consider the following average
\begin{equation}
\begin{split}
\langle e^{-i \Phi(t)} \rangle &= \langle \sum_{n=0}^{\infty} \dfrac{(-i )^n}{n!} \Phi(t)^n \rangle \\
&=  \sum_{n=0}^{\infty} \dfrac{(-i )^n}{n!}  \langle\Phi(t)^n \rangle \\
&=  \sum_{n=0}^{\infty} \dfrac{(-i )^n}{n!} m_n \\
&= \exp \left[ \sum_{n=1}^{\infty} \dfrac{(-i )^n}{n!} c_n \right]
\end{split}
\end{equation}
\end{document}
